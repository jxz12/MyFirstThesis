\chapter{Force-directed Layout}
we present advances in stress

\section{Background}
Because of this difference between the abstract mathematical structure and its representation within a visualisation, I will henceforth refer to the data itself as vertices and edges, and the representation as nodes and links.

introduce different visualisations for graphs (matrix, coin) lead into node-link
introduce types of node-link (like hive), lead into force-directed
introduce stress, demonstrate that stress and force are actually both optimisation but that force is gradient descent (and shouldn't be called force, more like velocity)
\subsection{Stress}
\url{http://deleeuwpdx.net/pubfolders/converge/converge.html}
admit that you have no idea about GEM 

\section{Stochastic Gradient Descent}
SGD
recommended cooling schedule changes for different things
\subsection{Results}
comparison to majorization
\subsection{Parameterisation}
finding $\eta_{\max}$ and $\varepsilon$.
\subsection{Large Graphs}
various bottlenecks
normal MDS community has some options like Halko et al. or GLINT, but an additional problem with graphs is that shortest paths still need to be computed, and so we cannot presume that the entire distance matrix is available to us.
multisource shortest paths is still used to find regions
but we present a simpler algorithm to find weights (make pseudocode)
note that it does not really minimise a particular function because the weights are asymmetric.

\section{Cookbook}
tweaks to the algorithm for various applications
\subsection{Radial Layout}
cool much more than usual for convergence
\subsection{Vertical Constraints}
should this go here or later?
\subsection{Regular Multidimensional Scaling}
show the digits dataset (note that weights are all set to 1)
try Phate classical MDS stuff
graphviz optimises in a subspace first?