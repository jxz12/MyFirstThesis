\chapter{Force-directed Node Layout}
If one were to ask a random group of people to draw a network on a piece of paper, it is likely that most would draw some dots to represent the nodes, and some lines joining the dots to represent the edges. This is a representation so intuitive that it is often synonymous with the abstract concept of a network entirely.
This is the reason why I have chosen to focus this thesis on this `join-the-dots' representation, henceforth known as a node-link diagram.
However it is useful to have an idea of other representations in order to gain a broader view of what network visualisation entails. A simple example of this includes the \emph{matrix plot}, which is simply a table where each vertex is represented by a row and a column, and each edge is a dot filled in at the intersection of a row and column.
Another example is the \emph{disc graph}, where nodes are represented by circles and edges exist if the circles overlap. This sees common use as a Venn diagram, but without the connotation of being a network.
A final, more obscure example is the \emph{string graph}, where each vertex is represented by a (possibly curved) line, and if they touch then 
See Figure~\ref{TODO}

In order to Because of this difference between the abstract mathematical structure and its representation within a visualisation, I will henceforth refer to the data itself as vertices and edges, and the representation as nodes and links.

introduce different visualisations for graphs (matrix, coin) lead into node-link
introduce types of node-link (like hive), lead into force-directed
This first technical chapter of the thesis will explore the problem of network layout, which is 

\section{Background}
introduce stress, demonstrate that stress and force are actually both optimisation but that force is gradient descent (and shouldn't be called force, more like velocity)
\subsection{Stress}
\url{http://deleeuwpdx.net/pubfolders/converge/converge.html}
admit that you have no idea about GEM 

\section{Stochastic Gradient Descent}
SGD
recommended cooling schedule changes for different things
\subsection{Results}
comparison to majorization
\subsection{Parameterisation}
finding $\eta_{\max}$ and $\varepsilon$.
\subsection{Large Graphs}
various bottlenecks
normal MDS community has some options like Halko et al. or GLINT, but an additional problem with graphs is that shortest paths still need to be computed, and so we cannot presume that the entire distance matrix is available to us.
multisource shortest paths is still used to find regions
but we present a simpler algorithm to find weights (make pseudocode)
note that it does not really minimise a particular function because the weights are asymmetric.

\section{Cookbook}
tweaks to the algorithm for various applications
\subsection{Radial Layout}
cool much more than usual for convergence
\subsection{Fixing one dimension}
adding extra initial iterations can help
\subsection{Regular Multidimensional Scaling}
show the digits dataset (note that weights are all set to 1)
try Phate classical MDS stuff
graphviz optimises in a subspace first?