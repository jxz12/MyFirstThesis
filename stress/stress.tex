\chapter{Force-directed Layout}
we present advances in stress

\section{Background}
introduce different visualisations for graphs, lead into node-link
introduce types of node-link (like hive), lead into force-directed
introduce stress, demonstrate that stress and force are actually both optimisation but that force is gradient descent (and shouldn't be called force, more like velocity)
\url{http://deleeuwpdx.net/pubfolders/converge/converge.html}
admit that you have no idea about GEM 

\section{Stochastic Gradient Descent}
SGD
recommended cooling schedule changes for different things
\subsection{Results}
comparison to majorization
\subsection{Parameterisation}
finding $\eta_{\max}$ and $\varepsilon$.
\subsection{Large Graphs}
various bottlenecks
multisource shortest paths is still used to find regions
but we present a simpler algorithm to find weights (make pseudocode)
note that it does not really minimise a particular function because the weights are asymmetric.

\section{Cookbook}
tweaks to the algorithm for various applications
\subsection{Radial Layout}
cool much more than usual for convergence
\subsection{Vertical Constraints}
should this go here or later?
\subsection{Regular Multidimensional Scaling}
show the digits dataset (note that weights are all set to 1)
try Phate classical MDS stuff