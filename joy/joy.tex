\chapter{EcoBuilder}

\section{Background}
application of SGD in an interactive setting

\subsection{Structure}
trophic levels
chain length
(mention loops)
\subsection{Dynamics}
May stability
feasibility

\section{Design and Hypotheses}
\subsection{Ap}
score
trophic levels

layout:
\begin{itemize}
    \item TROPHIC LEVELS:
    \begin{itemize}
        \item gauss-seidel iteration because positive definite, optimised to only take O(m) per iteration and O(n) extra space (because we only care about the sum of the rows and not where the rows are placed. In general we find the number of iterations required is <100 (test this) but requires for loops~\cite{oliviasimpsonpaper}
        \item whether this laplacian has a solution can also be found using a breadth first search from sources (which is also used for chain)
        \item note that this is the same set of equations as in \eqref{eq:tutte}
    \end{itemize}
    \item LAYOUT:
    \begin{itemize}
        \item SGD is so fast that we use it on every topological change
        \item 3D vs 2D: originally used 3D, but we found that players could not quickly pick up the rotational interface. Nodes would also often become obscured in the center of the layout, so we switched to 2D instead. 
        \item however this also made trophic level constraints more difficult to satisfy because there is one fewer dimension to move around in (previously all that was needed was to always keep the y-position. The best solution was to add 5 iterations of $\mu=1$ at the beginning, as an initialization step.
        \item localised majorization is used every frame for one node in order to fine-tune layout
        \item we compare constrained to non-constrained to see who performs better. They both use the same initial layout (SGD constrained to trophic levels) but with either constrained or non-constrained majorization when fine tuning.
        \item TODO: procrustes then performed on each connected component
    \end{itemize}
    
    \item superfocus
    \item johnson's algorithm for loops
    \item mention old idea for chess-board and why it didn't work
\end{itemize}

model:
\begin{itemize}
    \item scoring metric
    \item diagonals of interaction matrix (reference hsi-cheng's work)
    \item math.net library
\end{itemize}

UI:
\begin{itemize}
    \item database in sql
    \item indexed using...
\end{itemize}


\section{Results and Analysis}
