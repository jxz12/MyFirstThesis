\chapter{Hierarchical Edge Bundling}
The first chapter of this thesis was an exploration of how to position the nodes of a graph, and the natural question to then ask is how to deal with the only remaining component: the links. However this question is seemingly redundant at first, as the obvious answer is to simply draw straight lines between nodes with edges between them. While is by no means a poor choice, and is exactly how I have drawn all of the node-link diagrams thus far, this chapter will explore the possibility of drawing links using curves.

\section{Background}
The curving of edges in the concept of a node-link diagram is known as \textit{edge} bundling, and 

One of the most powerful visualisations is from Holten in hierarchical edge bundling. It is a visualisation that uses extra metadata. Here is the 
edge bundling and confluent drawing
blurring the line between the two (one is more practical, one is more theoretical)

clustering is a huge can of worms.
it is very important to realise that there are two types of clusters, which will be henceforth referred to as 'community' and 'betweenness' structure. This can also be referred to as 'assortative' and 'disassortative'.
some networks have both at the same time, and it is therefore difficult 
hierarchical clustering comparison and new distance measure(s)

Note that there is also a body of work on how to render links, once their trajectories are already known.

\section{Agglomerative Clustering}

mention the optimal ordering around a circle

\begin{itemize}
    \item girvan-newman -- this results in long strands that \item
\end{itemize}

\section{Power-confluent Drawings}
fixing issues and improving speed
b-spline algorithms and things